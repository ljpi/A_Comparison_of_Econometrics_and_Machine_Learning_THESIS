\documentclass[]{article}
\usepackage{lmodern}
\usepackage{amssymb,amsmath}
\usepackage{ifxetex,ifluatex}
\usepackage{fixltx2e} % provides \textsubscript
\ifnum 0\ifxetex 1\fi\ifluatex 1\fi=0 % if pdftex
  \usepackage[T1]{fontenc}
  \usepackage[utf8]{inputenc}
\else % if luatex or xelatex
  \ifxetex
    \usepackage{mathspec}
  \else
    \usepackage{fontspec}
  \fi
  \defaultfontfeatures{Ligatures=TeX,Scale=MatchLowercase}
\fi
% use upquote if available, for straight quotes in verbatim environments
\IfFileExists{upquote.sty}{\usepackage{upquote}}{}
% use microtype if available
\IfFileExists{microtype.sty}{%
\usepackage{microtype}
\UseMicrotypeSet[protrusion]{basicmath} % disable protrusion for tt fonts
}{}
\usepackage[margin=1in]{geometry}
\usepackage{hyperref}
\hypersetup{unicode=true,
            pdftitle={Machine Learning for Time Series Forecasting: A Comparison of New and Traditional Methods on Volatility},
            pdfauthor={Lester Pi; Faculty Advisor: Professor Randall R. Rojas},
            pdfborder={0 0 0},
            breaklinks=true}
\urlstyle{same}  % don't use monospace font for urls
\usepackage{color}
\usepackage{fancyvrb}
\newcommand{\VerbBar}{|}
\newcommand{\VERB}{\Verb[commandchars=\\\{\}]}
\DefineVerbatimEnvironment{Highlighting}{Verbatim}{commandchars=\\\{\}}
% Add ',fontsize=\small' for more characters per line
\usepackage{framed}
\definecolor{shadecolor}{RGB}{248,248,248}
\newenvironment{Shaded}{\begin{snugshade}}{\end{snugshade}}
\newcommand{\KeywordTok}[1]{\textcolor[rgb]{0.13,0.29,0.53}{\textbf{{#1}}}}
\newcommand{\DataTypeTok}[1]{\textcolor[rgb]{0.13,0.29,0.53}{{#1}}}
\newcommand{\DecValTok}[1]{\textcolor[rgb]{0.00,0.00,0.81}{{#1}}}
\newcommand{\BaseNTok}[1]{\textcolor[rgb]{0.00,0.00,0.81}{{#1}}}
\newcommand{\FloatTok}[1]{\textcolor[rgb]{0.00,0.00,0.81}{{#1}}}
\newcommand{\ConstantTok}[1]{\textcolor[rgb]{0.00,0.00,0.00}{{#1}}}
\newcommand{\CharTok}[1]{\textcolor[rgb]{0.31,0.60,0.02}{{#1}}}
\newcommand{\SpecialCharTok}[1]{\textcolor[rgb]{0.00,0.00,0.00}{{#1}}}
\newcommand{\StringTok}[1]{\textcolor[rgb]{0.31,0.60,0.02}{{#1}}}
\newcommand{\VerbatimStringTok}[1]{\textcolor[rgb]{0.31,0.60,0.02}{{#1}}}
\newcommand{\SpecialStringTok}[1]{\textcolor[rgb]{0.31,0.60,0.02}{{#1}}}
\newcommand{\ImportTok}[1]{{#1}}
\newcommand{\CommentTok}[1]{\textcolor[rgb]{0.56,0.35,0.01}{\textit{{#1}}}}
\newcommand{\DocumentationTok}[1]{\textcolor[rgb]{0.56,0.35,0.01}{\textbf{\textit{{#1}}}}}
\newcommand{\AnnotationTok}[1]{\textcolor[rgb]{0.56,0.35,0.01}{\textbf{\textit{{#1}}}}}
\newcommand{\CommentVarTok}[1]{\textcolor[rgb]{0.56,0.35,0.01}{\textbf{\textit{{#1}}}}}
\newcommand{\OtherTok}[1]{\textcolor[rgb]{0.56,0.35,0.01}{{#1}}}
\newcommand{\FunctionTok}[1]{\textcolor[rgb]{0.00,0.00,0.00}{{#1}}}
\newcommand{\VariableTok}[1]{\textcolor[rgb]{0.00,0.00,0.00}{{#1}}}
\newcommand{\ControlFlowTok}[1]{\textcolor[rgb]{0.13,0.29,0.53}{\textbf{{#1}}}}
\newcommand{\OperatorTok}[1]{\textcolor[rgb]{0.81,0.36,0.00}{\textbf{{#1}}}}
\newcommand{\BuiltInTok}[1]{{#1}}
\newcommand{\ExtensionTok}[1]{{#1}}
\newcommand{\PreprocessorTok}[1]{\textcolor[rgb]{0.56,0.35,0.01}{\textit{{#1}}}}
\newcommand{\AttributeTok}[1]{\textcolor[rgb]{0.77,0.63,0.00}{{#1}}}
\newcommand{\RegionMarkerTok}[1]{{#1}}
\newcommand{\InformationTok}[1]{\textcolor[rgb]{0.56,0.35,0.01}{\textbf{\textit{{#1}}}}}
\newcommand{\WarningTok}[1]{\textcolor[rgb]{0.56,0.35,0.01}{\textbf{\textit{{#1}}}}}
\newcommand{\AlertTok}[1]{\textcolor[rgb]{0.94,0.16,0.16}{{#1}}}
\newcommand{\ErrorTok}[1]{\textcolor[rgb]{0.64,0.00,0.00}{\textbf{{#1}}}}
\newcommand{\NormalTok}[1]{{#1}}
\usepackage{graphicx,grffile}
\makeatletter
\def\maxwidth{\ifdim\Gin@nat@width>\linewidth\linewidth\else\Gin@nat@width\fi}
\def\maxheight{\ifdim\Gin@nat@height>\textheight\textheight\else\Gin@nat@height\fi}
\makeatother
% Scale images if necessary, so that they will not overflow the page
% margins by default, and it is still possible to overwrite the defaults
% using explicit options in \includegraphics[width, height, ...]{}
\setkeys{Gin}{width=\maxwidth,height=\maxheight,keepaspectratio}
\IfFileExists{parskip.sty}{%
\usepackage{parskip}
}{% else
\setlength{\parindent}{0pt}
\setlength{\parskip}{6pt plus 2pt minus 1pt}
}
\setlength{\emergencystretch}{3em}  % prevent overfull lines
\providecommand{\tightlist}{%
  \setlength{\itemsep}{0pt}\setlength{\parskip}{0pt}}
\setcounter{secnumdepth}{0}
% Redefines (sub)paragraphs to behave more like sections
\ifx\paragraph\undefined\else
\let\oldparagraph\paragraph
\renewcommand{\paragraph}[1]{\oldparagraph{#1}\mbox{}}
\fi
\ifx\subparagraph\undefined\else
\let\oldsubparagraph\subparagraph
\renewcommand{\subparagraph}[1]{\oldsubparagraph{#1}\mbox{}}
\fi

%%% Use protect on footnotes to avoid problems with footnotes in titles
\let\rmarkdownfootnote\footnote%
\def\footnote{\protect\rmarkdownfootnote}

%%% Change title format to be more compact
\usepackage{titling}

% Create subtitle command for use in maketitle
\newcommand{\subtitle}[1]{
  \posttitle{
    \begin{center}\large#1\end{center}
    }
}

\setlength{\droptitle}{-2em}
  \title{Machine Learning for Time Series Forecasting: A Comparison of New and
Traditional Methods on Volatility}
  \pretitle{\vspace{\droptitle}\centering\huge}
  \posttitle{\par}
  \author{Lester Pi \\ Faculty Advisor: Professor Randall R. Rojas}
  \preauthor{\centering\large\emph}
  \postauthor{\par}
  \predate{\centering\large\emph}
  \postdate{\par}
  \date{June 13, 2017}


\begin{document}
\maketitle

{
\setcounter{tocdepth}{2}
\tableofcontents
}
\begin{Shaded}
\begin{Highlighting}[]
\NormalTok{knitr::opts_chunk$}\KeywordTok{set}\NormalTok{(}\DataTypeTok{echo =} \OtherTok{TRUE}\NormalTok{)}
\end{Highlighting}
\end{Shaded}

\section{Abstract}\label{abstract}

Traditionally econometric models such as autoregressive (AR), moving
average (MA), and autoregressive moving average (ARMA) models are used
to model, analyze, and forecast time series. However, with the advent of
big data and exponential increases in computing power there have been
groundbreaking leaps in the field of machine learning. This paper will
focus on various time series forecasting by comparing different methods
on financial and economic data. The first time series that will be
tested is the S\&P 500 30-day volatility, a continuous linear time
series. The second time series that will be tested is a binary
classification time series of a recession indicator. The goal is to
either find a model that out performs our traditional econometric
methods. The new models that we will be examining are LASSO regressions,
decision trees as random forests, and artificial neural networks. We
measure performance through the mean absolute percentage error (MAPE),
the classification accuracy and confusion matricies, and practical
feasibility. After running through rigerous testing methods on our data
sets, we propose two new models that feasibly out perform ARIMA. For
linear regression forecasting on S\&P 500, we propose an artificial
neural network model. For recession classification, we purpose a
decision tree implemented through a random forests model.

\section{1. Introduction}\label{introduction}

\subsection{1.1 Motivation}\label{motivation}

This paper is motivated by the interdisiplinary studies of applied
economics, specifically the application of statistics, computer science,
and data science to the field of economics. Traditionally, econometrics
dominated the field of applied economics. However, as critisized by
Peter Swann {[}1{]}, econometrics often times provides dissapointing
results and calls for action for a larger variety of research methods.
Therefore, we would like to compare traditional econometric techniques
with machine learning to improve upon the econometric techniques as well
as provide an alternative when econometrics does not suffice.

\subsection{1.2 Data}\label{data}

The following is a list of all external data series used, a description
of each, and the data source.

\begin{enumerate}
\def\labelenumi{\arabic{enumi}.}
\tightlist
\item
  S\&P 500 - The Standard \& Poor's 500 Index is a daily time series
  representing a weighted index of companies selected by economists and
  is often used as leading indicator for of U.S. equities and a
  performance measure of the market. We use this to calculate our first
  target variable, the 30-day S\&P 500 volatility. Source: Yahoo Finance
\item
  U.S. Recession - A daily time series of U.S. recession binary dummies.
  A recession is categorized as negative economic growth for at least
  two consecutive quarters. We use this time series as our second target
  variable. Source: FRED
\item
  CBOE VIX - A daily time series index for future 30-day volatility of
  the S\&P 500 calculated by the Chicago Board Options Exchange (CBOE).
  Often refered to as ``The Fear Index''. Source: Yahoo Finance
\item
  U.S. Effective Federal Funds Rate - A daily time series for what is
  often refered to as ``the interest rate''. Represents the rate at
  which the Federal Reserve sets that banks can borrow from each other
  at. Source: FRED
\item
  U.S. Presidential Approval Ratings - Data originally adpated from the
  Gallup Poll that contains approval ratings, disapproval ratings, and
  unknown ratings for each president. The data is not in a time series
  format, but was converted to a daily time series of monthly averages.
  Source: UCSB {[}from Gallup Poll{]}
\end{enumerate}

Note that ``daily'' refers to daily observances for when the stock
market is open. Non-stock time series data has been adjusted.

The following is a list of all daily time series created from the
external data sources and their descriptions.

\begin{enumerate}
\def\labelenumi{\arabic{enumi}.}
\tightlist
\item
  S\&P 500 30-Day Volatility - A measure of how volatile the S\&P 500 is
  over a 30 day period. Our first target variable. Created by taking the
  30-day volatility of the S\&P 500.
\item
  S\&P 500 30-Day Volatility Lagged Values - Lagged values from L(1) to
  L(30) of the S\&P 500 30-Day Volatility.
\item
  S\&P 500 Lagged Values - Lagged values from L(1) to L(30) of the S\&P
  500.
\item
  CBOE VIX Lagged Values - Lagged values from L(1) to L(30) of the CBOE
  VIX.
\item
  U.S. Presidential Approval Rating Averages - The daily time series
  constructed from monthly averages of the external U.S. Presidential
  Approval Ratings.
\item
  U.S. Effective Federal Funds Rate Momentum - The momentum of the U.S.
  Effective Federal Funds Rate calculated from taking the daily
  difference of the series on itself.
\end{enumerate}

\subsection{1.3 Forecasting Methods}\label{forecasting-methods}

Traditionally, economics and finance have relied heavily on econometrics
for time series forecasting. Our goal is to see if we can out perform
our econometrics techniques with machine learning.

\subsubsection{1.3.1 Econometrics}\label{econometrics}

Basic econometrics revolves around linear regression of the form:

\begin{center} <br>
\includegraphics{formulas/linear_regression.jpg} <br>
\end{center}

\paragraph{1.3.1.1 ARIMA}\label{arima}

\subsubsection{1.3.2 Machine Learning}\label{machine-learning}

We choose three distinct machine learning techniques, each with their
own advantages and disadvantages.

\paragraph{1.3.2.1 LASSO Regressions}\label{lasso-regressions}

\paragraph{1.3.2.2 Random Forests}\label{random-forests}

\paragraph{1.3.2.3 Artificial Neural
Networks}\label{artificial-neural-networks}

\section{2. S\&P 500 Volatility
Forecasting}\label{sp-500-volatility-forecasting}

\section{3. U.S. Recession
Forecasting}\label{u.s.-recession-forecasting}


\end{document}
